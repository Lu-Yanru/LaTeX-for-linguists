\documentclass{beamer}

%\usepackage{pgfpages}
%\mode<handout>{%
   % \pgfpagesuselayout{6 on 1}[a4paper] 
  %  \setbeameroption{show notes}
%}

% \usepackage{beamerthemesplit} %// Activate for custom appearance
\usetheme[compress]{Berlin}
\usecolortheme{dolphin}

\usepackage{etoolbox}

\usepackage{xeCJK}
\usepackage{xpinyin}
\xpinyinsetup{pysep = {}}

%suppress subsection line
\setbeamertemplate{headline}
{%
  \begin{beamercolorbox}[colsep=1.5pt]{upper separation line head}
  \end{beamercolorbox}
  \begin{beamercolorbox}{section in head/foot}
    \vskip2pt\insertnavigation{\paperwidth}\vskip2pt
  \end{beamercolorbox}%
  \begin{beamercolorbox}[colsep=1.5pt]{lower separation line head}
  \end{beamercolorbox}
}

%footer
\makeatother
\setbeamertemplate{footline}
{
  \leavevmode%
  \hbox{%
  \begin{beamercolorbox}[wd=.3\paperwidth,ht=2.25ex,dp=1ex,center]{author in head/foot}%
    \usebeamerfont{author in head/foot}\insertshortauthor
  \end{beamercolorbox}%
  \begin{beamercolorbox}[wd=.4\paperwidth,ht=2.25ex,dp=1ex,center]{title in head/foot}%
    \usebeamerfont{title in head/foot}\insertshorttitle
  \end{beamercolorbox}%
    \begin{beamercolorbox}[wd=.3\paperwidth,ht=2.25ex,dp=1ex,center]{date in head/foot}%
    \usebeamerfont{date in head/foot}\insertshortdate\hspace*{3em}
    \insertframenumber{} / \inserttotalframenumber\hspace*{1ex}
  \end{beamercolorbox}}%
  \vskip0pt%
}
\makeatletter
\setbeamertemplate{navigation symbols}{}
\setbeamertemplate{caption}[numbered]

\usepackage{xcolor}
\usepackage{url}
\usepackage{multicol}
\usepackage{hyperref}
\usepackage{ulem}
\normalem

\usepackage{graphicx}
\renewcommand\figurename{Abb.}
\renewcommand\tablename{Tab.}
\usepackage{float}
\usepackage{caption}
\usepackage{multirow}
% contents in tabular centers vertically
\usepackage{array}
% force line break in tabular cell
\newcommand{\tabincell}[2]{\begin{tabular}{@{}#1@{}}#2\end{tabular}}


%\usepackage{linguex}
\usepackage{gb4e}
\usepackage{cgloss4e}
% add space btw. lines
\noautomath
\setlength{\glossglue}{5pt plus 2pt minus 1pt}
\renewcommand{\eachwordthree}{\rule[-10pt]{0pt}{0pt}}
\let\eachwordone=\sffamily %or \normalfont
\let\eachwordtwo=\sffamily

\renewcommand{\@subex}[2]{\settowidth{\labelwidth}{#1}\itemindent\z@\labelsep#2%
         \ifnum\the\@xnumdepth=1%
           \topsep 7\p@ plus2\p@ minus3\p@\itemsep3\p@ plus2\p@\else%
           \topsep1.5\p@ plus\p@\itemsep1.5\p@ plus\p@%
           \parsep\p@ plus.5\p@ minus.5\p@%
           \advance\leftmargin#2\relax\fi}

\usepackage{ amssymb }


\title[short title]{Title}
\subtitle{Subtitle}
\author{author}
\institute{
	Email}

% logo
%\pgfdeclareimage[height=0.7cm]{480px-Huberlin-logo.pdf}
%\logo{
	%\vspace*{-0.25cm}
%	\includegraphics[width=2cm]{480px-Huberlin-logo.png}
	%\hspace*{-0.05cm}}
% default position	
%\titlegraphic{\includegraphics[width=2cm]{480px-Huberlin-logo.png}}
% specify position
\titlegraphic{%
  \begin{picture}(0,0)
    \put(160,5){\makebox(0,0)[rt]{\includegraphics[width=2cm]{uni-logo.png}}}
  \end{picture}}


%adjust font size to frame
%\usepackage{xpatch}

%\makeatletter
%\patchcmd\beamer@@tmpl@frametitle{\insertframetitle}{\insertsection\space-- \insertframetitle}{}{}
%\makeatother

%display short title
%\usepackage{etoolbox}
%\makeatletter
% Insert [short title] for \section in ToC
%\patchcmd{\beamer@section}{{#2}{\the\c@page}}{{#1}{\the\c@page}}{}{}
% Insert [short title] for \section in Navigation
%\patchcmd{\beamer@section}{{\the\c@section}{\secname}}{{\the\c@section}{#1}}{}{}
% Insert [short title] for \subsection in ToC
%\patchcmd{\beamer@subsection}{{#2}{\the\c@page}}{{#1}{\the\c@page}}{}{}
% Insert [short title] for \subsection  in Navigation
%\patchcmd{\beamer@subsection}{{\the\c@subsection}{#2}}{{\the\c@subsection}{#1}}{}{}
%\makeatother


\date{date}

\begin{document}

\frame{\titlepage
	\scriptsize{
	Modul\\
	SE\\
	Dozent\\
	Sprach- und literaturwissenschaftliche Fakultät\\
	Humboldt-Universität zu Berlin}}



\section{Einleitung}
\subsection{Motivation}
\frame
{
  \frametitle{Motivation}

  \begin{itemize}
  \item<1-> 
  \item<1-> 
  \end{itemize}
}

\subsection{Fragestellung}
\frame
{
  \frametitle{Fragestellung}

  \begin{itemize}
  \item<1-> Frage?     
  \end{itemize}
}

\subsection*{Gliederung}
\frame{
\frametitle{Gliederung}
\begin{multicols}{2}
\tableofcontents%[hideallsubsections]
\end{multicols}
}

\section{Wortstellung kontrastiv}
\frame
{
  \frametitle{Wortstellung des Deutschen, Englischen und Chinesischen kontrastiv}
\tableofcontents[currentsection, hideothersubsections]

}

\subsection{SVO}
\frame
{
  \frametitle{SVO im einfachen deklarativen Satz}
\begin{tabular}{| >{\raggedright\arraybackslash}m{0.15\linewidth} | >{\raggedright\arraybackslash}m{0.75\linewidth} |}
\hline
\tabincell{l}{Deutsch\\
\tiny{(Hoberg \&}\\\tiny{Hoberg, 2011)}}
& 
\begin{exe}
\ex \uline{Ich} {\color{blue}esse} \uuline{einen Apfel}.
\end{exe}

\\
\hline
\tabincell{l}{Englisch\\
\tiny{(Aarts, 2011)}}
&
\begin{exe}
\ex \uline{I} {\color{blue}eat} \uuline{an apple}.
\end{exe}

\\ \hline
\tabincell{l}{Chinesisch\\
\tiny{(Yip \&}\\\tiny{Rimmington, 2004)}}
&
\begin{exe}
\ex \gll \uline{\pinyin{wo3}} {\color{blue}\pinyin{chi1}} \uuline{\pinyin{ping2guo3}}.\\
	Ich ess- Apfel.\\
\end{exe}\\
\hline
\end{tabular}
}

\section{Empirische Analyse}
\frame
{
  \frametitle{Empirische Analyse}
\tableofcontents[currentsection, hideothersubsections]

}


\subsection{Transfer vom Chinesischen/Englischen}
\frame
{
  \frametitle{Transfer vom Chinesischen/Englischen}
\begin{itemize}
\item Kontaktstellung
\begin{exe}
\ex 
\glll 
 * \uline{Ich} {{\color{blue}muß}} {{\color{blue}fahren}} {in die Stadt.\footnote{Wei, 2006, p.100}}\\
 {} \uline{I} {{\color{blue}must}} {{\color{blue}go} (by car/bus)} {to the city.}\\
 {} \uline{\pinyin{wo3}} {{\color{blue}\pinyin{bi4xu1}}} {\pinyin{(zuo4 che1)} {\color{blue}\pinyin{qu4}}} \pinyin{cheng2li3.}\\
	
\end{exe}
\end{itemize}
}


\section{Diskussion}
\frame
{
  \frametitle{Diskussion}
  
  \begin{itemize}
  \item<1-> 
  \item<2-> 
   \end{itemize}
}

\begingroup
\scriptsize
\section{Literaturverzeichnis}
\frame
{
\begin{itemize}
\item 


\end{itemize}
}


\endgroup


\end{document}
